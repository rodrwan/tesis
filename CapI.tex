\chapter[Introducción]{Introducción}\label{ch:capitulo1}
%\fpar
En este capítulo presentaremos los argumentos base del problema que estamos proponiendo, con el fin de dar a conocer las principales razones que nos motivan a realizar este trabajo, además de mostrar en qué áreas se puede utilizar o se utiliza lo que proponemos, también presentaremos los objetivos generales y específicos de nuestra investigación y entregaremos una visión general de qué es lo que se podrá encontrar en los demás capítulos.

% ANTECEDENTES Y MOTIVACION

\section{Motivación}\label{chsub:Motivación}

Como consecuencia de las nuevas tecnologías de comunicación y al uso masivo de Internet en la sociedad actual, la cantidad de información audiovisual disponible en formato digital está alcanzando cifras realmente elevadas. Es por ese motivo que ha sido preciso diseñar sistemas que nos permitan describir el contenido de varios tipos de información multimedia para poderlos buscar y clasificar.

Hoy en día y con los avances tecnológicos realizados a la fecha, es inevitable no querer abordar desafíos relacionados con la detección y reconocimiento de objetos o personas. Algunos ejemplos los podemos ver en las propuestas realizadas por \textit{Google}, el que con tan solo subir una imagen nos entrega un sin fin de imágenes que tienen relación con la imagen de entrada, o \textit{Facebook}, el cual detecta automáticamente los rostros de personas, y ni pensar lo que pueden llegar a hacer. Quizás en un futuro, incluso detecten que estás en cierto lugar, con tan solo analizar el contenido de una imagen en la que apareces. Incluso sistemas de reconocimiento de patentes de vehículos, como lo que presenta \textit{Sutec}, el cual analiza vídeos con el fin de poder detectar dichos elementos lo que puede permitir identificar automóviles robados. Inclusive ya desde hace algún tiempo, las cámaras digitales son capaces de reconocer cuando alguien sonríe y tomar una foto, así mismo los teléfonos inteligentes son capaces de modificar las imágenes capturadas, agregándoles filtros que hacer que parezcan fotografías hechas por un profesional. SMOWL es una solución biométrica que emplea un algoritmo automático para verificar o reconocer la identidad de una persona en función de sus características fisiológicas. En general, un sistema de identificación biométrico hace uso de cualquiera de las características fisiológicas (por ejemplo, una huella digital, patrón del iris, o cara) o patrones de comportamiento (tales como, la escritura a mano, la voz, o el tecleado) para identificar a una persona~\cite{SMOWL}.

Diversas son los posibilidades de desarrollar trabajos en el área de la detección y reconocimiento de elementos en imágenes y vídeos. Es así que páginas web como \textit{Flickr}, \textit{Picasa}, \textit{Instagram}, entre muchos otros nos permiten obtener una cantidad inimaginable de datos para estudiar, con lo cual se pueden crear aplicaciones que puedan reconstruir lugares geográficos con solo imágenes disponibles en redes sociales. Poder comparar productos con tan solo detectarlos con un teléfono inteligente. Poder hacer comparaciones en tiempo real de lugares geográficos, tomando una foto y comparando esta con un sin número de otras imágenes de ese mismo sitio y así poder detectar cambios en el patrimonio cultural. Detectar a personas que entran a un edificio y saber si son conocidos de alguno de los habitantes del lugar. Detectar un rostro y por ejemplo poder saber cual es su estado de ánimo, saber en detalle si está asustado o tiene pena, o si es que algo le causó alegría, poder saber que sensaciones en base a las expresiones de su rostro está teniendo. En áreas de la medicina, poder detectar patrones moleculares automáticamente en una muestra digital. Es tanto y tan variado lo que se puede hacer con el análisis de imágenes que lo único que nos limita es la imaginación, basta con tan solo salir a dar un pequeño paseo y darse cuenta que inventos como: \textit{Google} \textit{Glass}, teléfonos inteligentes, tablets, entre muchos otros inventos que aparecerán en un futuro, nos abrirán las puertas para que el estudio de las imágenes sea un fuente de trabajos casi infinita. Es por esto que tener la posibilidad de poder competir con grandes de la tecnología nos motivan a crear y aventurarnos a realizar nuestra propia propuesta para hacer cosas novedosas y que llamen la atención de las personas.

% frase remate
Es por esto que nuestra principal motivación es poder generar un descriptor (Definición~\ref{def:desc}) eficiente, en tiempo de ejecución, el que sea capaz de representar un objetos en particular de forma precisas y que además sea ligero, para poder ser utilizado en sistemas donde los recursos sean escasos. Además con nuestros hallazgos, aportar nuevos conocimientos y nuevas opciones al estado del arte. La importancia de este estudio es poder generar contenido que sea utilizado posteriormente como base para nuevas investigaciones y desarrollos tanto en la industria, como en la educación. Nos motiva también el tener la posibilidad de poder publicar nuestro estudio, con el objetivo de poder ser un aporte a la ciencia actual.

\section{Contexto}\label{sub:Contexto}

Hoy en día es habitual escuchar a las personas comentar sobre las cosas que hace Facebook o Google en el campo de visión por computador, por ejemplo, Facebook reconoce los rostros de las personas y te recomienda etiquetarlos o te dice ``esta persona puede ser conocido tuyo'' por medio de una fotografía, ó como ``Google Image'' que con tan solo escribir una palabra, ésta te entrega un conjunto de imágenes extraídas de la web relacionadas con la palabra escrita, estos ejemplos despertaron gran interés en nosotros, por lo que nos adentramos a explorar esta área de la informática.

Presentaremos una manera eficiente, ligera y rápida para la identificación de objetos y expresiones faciales. El objetivo principal es que nuestra propuesta sea los suficientemente ligera para poder ser utilizada en dispositivos móviles, la finalidad, por ejemplo, es pasear por un museo y dentro de este, visualizar un elemento; y mientras lo observamos obtenemos información sobre este mismo de forma rápida y precisa, incluso poder ir por la calle, ver a personas y poder saber quienes son antes de que nos saluden, detectar objetos interesantes y obtener información sobre ellos, llevar la computación más allá del viejo y olvidado ordenador.

Nos centraremos en la investigación de un nuevo mecanismo de descripción de características con el fin de crear una estructura que represente eficazmente dichas características y sea ligero para su procesamiento en el reconocimiento de objetos y expresiones.

\section{Problema}\label{sec:problema}

Si bien a la fecha estudios como los realizados por P. Felzenszwalb, D. Ramanan, A. Ramirez, entre otros grandes investigadores, nos dan una clara idea de que se pueden hacer muchas cosas en el campo de visión por computadora, análisis de imágenes y análisis de patrones, es vital avanzar en estudios relacionados con estas áreas, ya que con la aparición de nuevas tecnologías y tendencias se exigen avances tecnológicos acordes a los tiempos en los cuales estamos viviendo. Junto a esto y con el masivo uso de los teléfonos inteligentes, redes sociales, y medios de comunicación digital, se genera información a cada momento, información que puede ser procesada y utilizada para tomar decisiones, incluso para crear nuevas aplicaciones. Pero, para poder lograr esto es que se necesitan nuevos algoritmos que puedan ser rápidos y eficientes, los cuales puedan ser portables a casi cualquier dispositivo y por supuesto que sean adaptables a cualquier tipo de situación. Gracias a estos avances tecnológicos surgen diversos problemas o nuevas necesidades, que de cierta forma deben ser satisfechas. Por lo que nosotros afrontaremos el problema de crear un algoritmo el que sea capaz de extraer las características más relevantes de un conjunto de imágenes, con el fin de poder representar de manera rápida y con la menor pérdida de información para ser utilizado en tareas de detección y reconocimiento de objetos y expresiones faciales. Es natural que preguntas tales como, ¿qué tan rápido es nuestro algoritmo?, ¿qué tan buena es la representar obtenida para un objetos o expresión?, ¿será lo suficientemente ligero y rápido para ser utilizado en dispositivos móviles? o ¿entregará los resultados esperados? serán respondidas a lo largo de esta investigación.

\section{Objetivos}
La idea principal es generar un descriptor basado en partes utilizando métodos de reconocimiento de patrones temporales.
\begin{itemize}
		\item Objetivos generales:
			\begin{itemize}
				\item Combinar los diferentes métodos para generar un nuevo descriptor.
				\item Analizar técnicas basadas en partes y basadas en apariencia, tanto para objetos como para expresiones faciales en secuencias de imágenes.
			\end{itemize}
		\item Objetivos específicos:
			\begin{itemize}
				\item Analizar y evaluar el comportamiento del descriptor en conjuntos de imágenes, para poder obtener métricas que nos permitan identificar que tan buena es nuestra propuesta respecto a la identificación de objetos.
				\item Analizar y evaluar el comportamiento del descriptor en diversas imágenes, para poder obtener métricas que nos permitan identificar que tan buena es nuestra propuesta con expresiones faciales.
				\item Analizar la precisión del descriptor, es decir, obtener métricas que nos permitan verificar mediante experimentos que tan eficiente es el algoritmo propuesto.
				\item Analizar el rendimiento y velocidad de cómputo de nuestra propuesta con métricas específicas para analizar dicho comportamiento.
				
			\end{itemize}			

	\end{itemize}

\section{Solución propuesta}

Nuestro trabajo propone mejorar el trabajo de Felzenszwalb et al.~\cite{Felzenszwalb2008,Felzenszwalb2010,Felzenszwalb2013}, utilizando como medio el framework creado por el grupo de trabajo de D. Ramanan. Esto nos ayudará a crear y probar nuestro algoritmo, con el fin de hacer una comparación entre lo presentado en el estado del arte (ver Capítulo~\ref{ch:capitulo2}). En concreto, crearemos un algoritmo para representar objetos reales de tal forma que este proceso sea más rápido y eficiente en el sentido de obtener mejores resultados que lo presentado por Felzenszwalb et al. Además utilizaremos esta implementación para poder clasificar rostros con distintas expresiones faciales.

\section{Resultados esperados}
Esperamos poder crear un nuevo descriptor robusto con ayuda de los trabajos hechos con anterioridad relacionados con este tema para poder identificar tanto objetos como expresiones. Diseñaremos nuestros algoritmos con el fin de generar un descriptor eficiente (con la idea de que pueda ser utilizado en un sistema móvil), para así ayudar a las personas a obtener información del mundo real de forma más fácil y dinámica.

A su vez queremos analizar y verificar que nuestra propuesta cumple con los estándares, que presentaremos en el estado del arte de nuestra investigación (ver Capítulo~\ref{ch:capitulo2}), para así proponer nuestro método como una solución robusta y eficiente en comparación a lo ya realizado por otras investigaciones que existen en la actualidad.

\section{Resumen}

Luego de haber presentado en este capítulo los antecedentes previos, tanto lo que nos motiva, como el contexto en el que estamos trabajando para solucionar el problema que planteamos, en los siguientes capítulos se presentan, los trabajos relacionados (Capítulo~\ref{ch:capitulo2}) lo que nos permitirá tener una base sólida para poder generar un propuesta mejor que la que se puede encontrar en el estado del arte, estos trabajos nos sirven de guía para el desarrollo de lo que finalmente deseamos solucionar. 

Algunas definiciones preliminares serán descritas en el Capítulo~\ref{ch:capitulo3} las que nos ayudarán a entender y comprender ciertos conceptos básicos que serán utilizados en el Capítulo~\ref{ch:capitulo4}, donde se presentará el framework construido por el grupo de trabajo de D. Ramanan, el que nos permitirá probar nuestra propuesta de algoritmo. Posteriormente, en el Capítulo~\ref{ch:capitulo5} presentaremos los experimentos y los resultados obtenidos tanto para objetos como para expresiones faciales. Finalmente, el Capítulo~\ref{ch:capitulo6} abre el espacio para discutir y reflexionar acerca de la solución al problema planteado en el Capitulo~\ref{ch:capitulo1}, como también se presentan los desafíos futuros.