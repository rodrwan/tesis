\chapter[Introducción]{Introducción}\label{ch:capitulo1}
%\fpar
En este capítulo pasaremos a revisar las bases del problema que estamos planteando, con el fin de dar a conocer nuestra motivación, el contexto del problema y a su vez presentar los objetivos generales y específicos de nuestra investigación.

% ANTECEDENTES Y MOTIVACION

\section{Motivación}\label{chsub:Motivación}

Nuestra principal motivación es poder generar un descriptor eficiente, en tiempo de ejecución y que sea ligero, además de aportar nuevos conocimientos y nuevas opciones al estado del arte actual. La importancia de este estudio es poder generar contenido que sea utilizado posteriormente como base para nuevas investigaciones. En detalle, nuestra propuesta nos motiva a preparar un estudio digno de ser publicado y usado como referencia. Además, hoy en día y con los avances tecnológicos realizados a la fecha, es inevitable no querer abordar desafíos relacionados con la detección y reconocimiento de objetos y personas. Algunos ejemplos los podemos ver en las propuestas realizadas por Google, el que con tan solo subir una imagen, nos entrega un sin fin de imágenes que tienen relación con la imagen de entrada, o Facebook el cual detecta automáticamente los rostros de personas, y ni pensar lo que pueden llegar a hacer. Quizás en un futuro, incluso detecten que estás en cierto lugar, con tan solo analizar el contenido de una imagen en la que apareces. Diversos son los posibles trabajos que se pueden hacer en esta área. Es así como estos dos grandes de la tecnología nos motiva a crear nuestra propia propuesta para hacer cosas novedosas y que llamen la atención de las personas.

\section{Problema}\label{sec:problema}
Si bien 
\section{Contexto}\label{sub:Contexto}

Hoy en día es habitual escuchar a las personas comentar sobre las cosas que hace Facebook o Google en el campo de visión por computador, por ejemplo, Facebook reconoce los rostros de las personas y te recomienda etiquetarlos o te dice ``esta persona puede ser conocido tuyo'' por medio de una fotografía, ó como ``Google Image'' que con tan solo escribir una palabra, éste te entrega un conjunto de imágenes extraídas de la web relacionadas con la palabra escrita, estos ejemplos despertaron gran interés en nosotros, por lo que nos aventuramos a explorar esta área de la informática.

Presentaremos una manera eficiente, ligera y rápida para la identificación de objetos y expresiones faciales. El objetivo principal es que nuestra propuesta sea los suficientemente ligera para poder ser utilizada en dispositivos móviles, la finalidad, por ejemplo, es pasear por un museo y dentro de este, visualizar un elemento; y mientras lo observamos obtenemos información sobre este mismo de forma rápida y precisa, incluso poder ir por la calle, ver a personas y poder saber quienes son antes de que nos saluden, detectar objetos interesantes y obtener información sobre ellos, llevar la computación más allá del viejo y olvidado ordenador.

Nos centraremos en la investigación de un nuevo mecanismo de descripción de características con el fin de crear una estructura que represente eficazmente dichas características y sea ligero para su procesamiento en el reconocimiento de objetos y expresiones.

\section{Objetivos}
La idea principal es generar un descriptor basado en partes utilizando métodos de reconocimiento de patrones temporales.
\begin{itemize}
		\item Objetivos generales:
			\begin{itemize}
				\item Combinar los diferentes métodos para generar un nuevo descriptor.
				\item Analizar técnicas basadas en partes y basadas en apariencia, tanto para objetos como para expresiones faciales en secuencia de imágenes.
			\end{itemize}
		\item Objetivos específicos:
			\begin{itemize}
				\item Analizar y evaluar el comportamiento del descriptor con conjuntos de imágenes, para poder obtener métricas que nos permitan identificar que tan buena es nuestra propuesta para identificar objetos.
				\item Analizar y evaluar el comportamiento del descriptor con diversas imágenes, para poder obtener métricas que nos permitan identificar que tan buena es nuestra propuesta con expresiones faciales.
				\item Analizar la precisión del descriptor, es decir, obtener métricas que nos permitan verificar mediante experimentos que tan eficiente es el algoritmo propuesto.
				\item Analizar el rendimiento y velocidad de computo de nuestra propuesta con métricas especificas para analizar dicho comportamiento.
				
			\end{itemize}			

	\end{itemize}

\section{Solución propuesta}

Nuestro trabajo propone mejorar el trabajo de Felzenszwalb et al.~\cite{Felzenszwalb2010}, utilizando como medio el marco de desarrollo creado por el grupo de trabajo de D. Ramanan. Esto nos ayudará a crear y probar nuestro algoritmo, con el fin de hacer una comparación entre lo presentado en el estado del arte (ver Sección~\ref{ch:capitulo2}) por Felzenszwalb. En concreto, crearemos un algoritmo para representar objetos reales de tal forma que este proceso sea más rápido y eficiente en el sentido de obtener mejores resultados que lo presentado en Felzenszwalb et al.~\cite{Felzenszwalb2010}.

\section{Resultados esperados}
Esperamos poder crear un nuevo descriptor robusto con ayuda de los trabajos hechos con anterioridad relacionados con este tema para poder identificar tanto objetos como expresiones. Diseñaremos nuestros algoritmos con el fin de generar un descriptor eficiente (con la idea de que pueda ser utilizado en un sistema móvil), para así ayudar a las personas a obtener información del mundo real de forma más fácil y dinámica.

A su vez queremos analizar y verificar que nuestra propuesta cumple con los estándares vistos en el estado del arte de nuestra investigación, para así proponer nuestro método como, un método robusto y eficiente en comparación con los que existen en la actualidad.

\section{Resumen}

